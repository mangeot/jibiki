%% Exemple de source LaTeX pour un article soumis à TALN
\documentclass[10pt,a4paper,twoside]{article}

\usepackage[utf8]{inputenc}
\usepackage[T1]{fontenc}
\usepackage{graphicx}

% Package utilisé uniquement pour l'exemple.
\usepackage{lipsum}
\usepackage{hyperref}

% faire les \usepackage dont vous avez besoin AVANT le \usepackage{jeptaln2012} 
% add the \usepackage for you packages BEFORE the \usepackage{jeptaln2012}

\usepackage{talnrecital2013}
% Insérer les définitions de biblio en français (cf apalike-fr.bst)
\usepackage[frenchb]{babel}

\title{Bases lexicales multilingues : traitement des acronymes}

\author{Ying Zhang\up{1}\quad Mathieu Mangeot\up{1}\\
{\small  (1) GETALP-LIG, 41 rue des mathématiques  BP 53 38041 Grenoble Cedex 9\\ 
  \texttt{ying.zhang@imag.fr, mathieu.mangeot@imag.fr} \\ 
}}

\begin{document}

\maketitle

%% In an English article, use \resumeEn with 2 arguments (french title and french summary)
\resume{
Ici, un résumé en français (max. 150 mots).\\
\lipsum[1]
}

%% In an English article, use \abstractEn with 1 arguments (english summary)
\abstract{TALN-RECITAL2013 template (English translation of the article title)}{
The translation of the title in English is mandatory to enhance visibility of the online version of the papers in international scientific article databases (DBLP, citeseer, etc.). Here an abstract in English (max. 150 words).\\
\lipsum[2]
}

\motsClefs{base lexicale multilingue, macrostructure, Jibiki, Pivax, Common Dictionary Markup, iPoLex, entrepôt de données linguistiques}
{multilingual lexical database, macrostructure, Jibiki, Pivax, Common Dictionary Markup, iPoLex, linguistic data warehouse}

%\motsClefs{Ici une liste de mots-clés en français}
%{Here a list of keywords in English}

%% Aller à la page suivante si nécessaire
%\newpage
%%================================================================
\section{Introduction}

\subsection{Situation}

L

\subsection{Intérêt}

L

\subsection{Présentation}

L

\section{Problématique}

% reprendre l'explication du cahier des charges des acronymes

L

\section{Données : choix de la macrostructure}

% dans cet article, on ne détaille pas la microstructure. On l'évoquera seulement

\subsection{Macrostructure Papillon}

% utilise le concept de lexies (TST, Mel'cuk)
% présente le concept de structure pivot et d'axie
L \cite{GSMM01a}

\subsection{Macrostructure Pivax}

% présente le concept d'axème, de multiples volumes dans la même langue et de structure pivot à étages
L \cite{MMHTN09}

\subsection{Macrostructure ProAxies}

% utilise les concepts présentés précédemment (structure pivot à étages, multiples volumes dans la même langue, lexies, axies, axemes)
% utilise le concept de prolexème (au même niveau que les axèmes)
% présente le concept de proaxie

L

\section{Outils nécessaires : plateformes de manipulation}

\subsection{Plateforme Jibiki v1}

% parle des différents outils existants : tswanelex, DPC, voir
\cite{MMCE11}
% présente la plateforme : CDM et table d'index, édition générique, opensource, licence GPLv3, dispo publiquement et gratuitement sur ligforge en SVN, sert pour de multiples projets
% explique les limitations : pas de liens entre plusieurs volumes différents, macrostructures complexes (pivot, pivax, etc.) codées en dur, difficultés de création de méta-données

\cite{MMAC06}

\subsection{Gestion des données et méta-données : iPoLex}

% explique le concept d'entrepôt de données
% génération assistée des méta-données

\subsection{Gestion des liens riches : extension Jibiki-Pivax}

% explique la nouvelle gestion des liens riches : 
% liens : volume de destination, poids, type (axi, final), langue, étiquette libre, etc. 
% table de liens séparée
% algorithme de collecte des liens dans DictionariesFactory.java (expandResults)
% algorithme de construction du résultat : parcours montant vers les axies puis descendant vers les lexies cibles

\section{Résultats préliminaires}

% montre les résultats de recherche pour différents scénarios

% scénario 1 : montre si une axie est dispo ONU => ONU en chinois 
% scénario 2 : montre s'il n'y a pas d'axie de dispo, on passe par la proaxie et on utilise l'étiquette (label) Organisation des Nations Unies => United Nations
% scénario 3 : montre s'il n'y a pas d'axie de dispo et pas d'étiquette correspondante. On utilise seulement la proaxie : Trouver un exemple avec un acronyme dans une langue et pas dans une autre

\section{Conclusion}

L

% à effacer à la fin
\section{Exemples à effacer à la fin}

\begin{itemize}
\item Une liste à puce
\item avec plusieurs lignes
\item pas trop espacées... 
\end{itemize}

\begin{enumerate}
\item Une liste numérotée
\item où le 2. succède, étrangement, au 1.
\item et le trois, au deux... (ce serait pas ambigu cela ?)
\end{enumerate}

\subsubsection{Figures et tables}

Les figures et les tables seront centrées sur la page avec une légende située en dessous. La légende contiendra le mot clé figure (ou table) en petite capitale, suivi du numéro de la figure ou de la table (numéros indépendants). La figure \ref{image} et la table \ref{table} en sont un exemple. Les équations peuvent figurer "en ligne" ou centrées sur la page, sans légende. Un numéro de renvoie peut figurer à droite de l'équation pour permettre les références dans le texte.

\begin{table}[!h]
\centering
	\begin{tabular}{|c|p{4cm}|}
	\hline
	Un tableau&\\
	\hline
	&Les cellules ainsi que le tableau sont centrés\\
	\hline
	\end{tabular}
\caption{Un tableau}\label{table}
\end{table}

\begin{figure}[htbp] 
\begin{center} 
\includegraphics{images/atala.png}
\end{center} 
\caption{Une image comme figure} \label{image} \
\end{figure}
Paragraphe facultatif

%%================================================================
%% Note : si l'on préfère éviter de factoriser les crossrefs :
%% bibtex -min-crossrefs 99 taln-exemple
%%================================================================

\bibliographystyle{apalike-fr}
\bibliography{biblio-mangeot}

%%================================================================
\end{document}
